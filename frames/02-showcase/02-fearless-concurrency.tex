\subsection{Fearless Concurrency}%

\begin{frame}{Data Races}%
	\centering%
	\begin{tikzpicture}
		\begin{scope}[outer sep = 5pt]
			\node (t1) {Thread 1};
			\node[right = 1.5cm of t1] (data) {Data};
			\node[right = 1.5cm of data] (t2) {Thread 2};
		\end{scope}

		\begin{scope}[every node/.style = { draw }, inner sep = 5pt]
			% Data
			\node[below = 0.5 of data] (d1) {10};
			\node[transparent, below = 0.5 of d1] (d2) {10}; % Placeholder

			\node[visible on = <5->, below = 0.5 of d2] (d3) {12};
			\draw[visible on = <-4>] (d3.north) -- (d3.south);

			\node[visible on = <7->, below = 0.5 of d3] (d4) {15};
			\draw[visible on = <-6>] (d4.north) -- (d4.south);

			% Threads
			\node[visible on = <2->] at (t1 |- d1) (buf1) {10};
			\draw[visible on = <1>] (buf1.north) -- (buf1.south);

			\node[visible on = <4->] at (t1 |- d3) (buf2) {12};
			\draw[visible on = <-3>] (buf2.north) -- (buf2.south);

			\node[visible on = <3->] at (t2 |- d2) (buf3) {10};
			\draw[visible on = <-2>] (buf3.north) -- (buf3.south);

			\node[visible on = <6->] at (t2 |- d4) (buf4) {15};
			\draw[visible on = <-5>] (buf4.north) -- (buf4.south);
		\end{scope}

		\path (t1) +(0, -5.5cm) coordinate (end);
		\draw[-Latex] (t1) -- (buf1) -- (buf2) -- (t1 |- end);
		\draw[-Latex] (t2) -- (buf3) -- (buf4) -- (t2 |- end);
		\draw[-Latex] (data) -- (d1) -- (d3) -- (d4) -- (d4 |- end);

		\draw[visible on = <2->, -Latex] (d1) -- (buf1);
		\draw[visible on = <5->, -Latex] (buf2) -- (d3);
		\draw[visible on = <3->, -Latex] (d1) -- ($(d1)!0.5!(d1 -| buf3)$) |- (buf3);
		\draw[visible on = <7->, -Latex] (buf4) -- (d4);

		\begin{scope}[outer sep = 3pt]
			\draw[visible on = <4->, -Latex] (buf1) -- +(-1cm, 0) |- node[pos = 0.25, left] {+2} (buf2);
			\draw[visible on = <6->, -Latex] (buf3) -- +(1cm, 0) |- node[pos = 0.25, right] {+5} (buf4);
		\end{scope}
	\end{tikzpicture}%
\end{frame}%

\defverbatim[colored]\immutableModification{%
	\begin{minted}{rust}
        fn add_value(v: &Vec<i32>) {
            v.push(42);
        }
    \end{minted}
}%
\defverbatim[colored]\mutableModification{%
	\begin{minted}{rust}
        fn add_value(v: &mut Vec<i32>) {
            v.push(42);
        }
    \end{minted}
}%
\defverbatim[colored]\useMutableModification{%
	\begin{minted}{rust}
        let mut v = vec![1, 2];
        add_value(&mut v);
    \end{minted}
}%
\begin{frame}{Explicit Mutability}%
	\only<1>{\immutableModification}%
	\only<2->{\mutableModification}%
	\nointerlineskip%
	\begin{overprint}%
		\onslide<1>\compilerError{0596}{cannot borrow {`*v`} as mutable, as it is behind a {`\&`} reference}%
		\onslide<3>\useMutableModification%
	\end{overprint}%
\end{frame}%

\defverbatim[colored]\readingLoop{%
	\begin{minted}{rust}
        let mut v = vec![1, 2];
        for e in &v {
            println!("{e}, {}", v[0]);
        }
    \end{minted}
}%
\defverbatim[colored]\writingLoop{%
	\begin{minted}{rust}
        for e in &v {
            v.push(*e);
        }
    \end{minted}
	\compilerError{0502}{cannot borrow {`v`} as mutable because it is also borrowed as immutable}%
}%
\defverbatim[colored]\extendFromWithin{%
	\begin{minted}{rust}
        v.extend_from_within(..);
    \end{minted}
}%
\begin{frame}{Mutability Control}%
	\readingLoop%
	\nointerlineskip%
	\begin{overprint}%
		\onslide<2>\writingLoop%
		\onslide<3>\extendFromWithin%
	\end{overprint}%
\end{frame}%

\defverbatim[colored]\threadByRef{%
	\begin{minted}{rust}
        let mut v = vec![1, 2];
        thread::spawn(|| v.push(3)).join().unwrap();
        println!("{}", v[2]);
    \end{minted}
	\compilerError{0373}{closure may outlive the current function, but it borrows {`v`}, which is owned by the current function}\\[1em]
	\compilerError{0502}{cannot borrow {`v`} as immutable because it is also borrowed as mutable}%
}%
\defverbatim[colored]\threadByRc{%
	\begin{minted}{rust}
        let v = Rc::new(RefCell::new(vec![1, 2]));
        thread::spawn({
            let v = v.clone();
            move || v.borrow_mut().push(3)
        })
        .join()
        .unwrap();
        println!("{}", v.borrow()[2]);
    \end{minted}
}%
\defverbatim[colored]\threadByArc{%
	\begin{minted}{rust}
        let v = Arc::new(Mutex::new(vec![1, 2]));
        thread::spawn({
            let v = v.clone();
            move || v.lock().unwrap().push(3)
        })
        .join()
        .unwrap();
        println!("{}", v.lock().unwrap()[2]);
    \end{minted}
}%
\begin{frame}{Threads}%
	\only<1>{\threadByRef}%
	\only<2>{\threadByRc}%
	\only<3>{\threadByArc}%

	\nointerlineskip\only<2->{%
		\begin{overprint}%
			\onslide<2>\compilerError{0277}{{`Rc<RefCell<Vec<i32>>>`} cannot be shared between threads safely}%
			\onslide<3>%
		\end{overprint}%
	}%
\end{frame}
